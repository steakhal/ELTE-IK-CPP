%%%%%% Basic packages begin %%%%%%
\usepackage[
	textwidth  = 160mm, 
	textheight = 230mm, 
	top        = 25mm, 
	bottom     = 30mm
]{geometry}
\usepackage[normalem]{ulem}
\usepackage[utf8]{inputenc}
\usepackage[T1]{fontenc}
\PassOptionsToPackage{defaults=hu-min}{magyar.ldf}
\usepackage[magyar]{babel}
%%%%%%% Basic packages end %%%%%%%

%%%%%% Packages required for this document begin %%%%%%
\usepackage{
	amsmath,   % Math mode
	amsthm,    % "note" environment
	amsfonts,  % "\mathbb{}" command
	paralist,  % "compactitem" and "compactenum" environment
	multirow,  % "\multirow{}{}" command
	float,     % "H" float specifier
	tikz,      % Basepackage for nearly all figures
	tikz-uml,  % Used for UML diagrams. Package needs to be in the same directory!
	listings,  % Used for code snippets
	subfiles,  % Used for including subfiles into this document
	array      % "\setlength{\extrarowheight}{2pt}"
}
\usepackage[unicode]{hyperref} % Clickable links
%%%%%%% Packages required for this document end %%%%%%%

%%%%%% TikZ options start %%%%%%
\usetikzlibrary{
	positioning, % Contains positioning utilities, such as "below = of" 
	calc,        % Adding coordinates
	math         % Needed for global variables
}

\tikzmath{
	\stackNodeWidth  = 20;
	\stackNodeHeight = 5;
}

\tikzstyle{NodeBase} = [
	rectangle,
	text centered,
	draw = black
]

\definecolor{DefaultObjectColor}{gray}{0.9} % This is the default color of object in the TikZ pictures

\tikzstyle{StackObject} = [
	NodeBase,
	minimum width  = \stackNodeWidth mm, 
	minimum height = \stackNodeHeight mm,
	fill           = DefaultObjectColor
]

\tikzstyle{BinTreeNode} = [
	NodeBase,
	minimum width  = 1cm, 
	minimum height = 5mm, 
	fill           = DefaultObjectColor
]

\tikzstyle{Heap} = [
	NodeBase,
	minimum width  = 5.5cm, 
	minimum height = 2.5cm,
	fill=white
]

\tikzstyle{SikidomFrame} = [
	NodeBase, 
	minimum width  = 2cm, 
	minimum height = 10mm, 
	line width     = 2pt
]

% Creates a stack frame in a TikZ environment
%
% param #1 x coordinate position
% param #2 y coordinate position
% param #3 height
\newcommand{\drawStackFrame}[3]{
	\draw [thick, black] (#1, #2) -- (#1 + \stackNodeWidth/10, #2);
	\draw [thick, black] (#1, #2) -- (#1, #3);
	\draw [thick, black] (#1 + \stackNodeWidth/10, #2) -- (#1 + \stackNodeWidth/10, #3);
}

\tikzstyle{arrow} = [
	thick,
	->,
	>=stealth
]
%%%%%%% TikZ options end %%%%%%%

%%%%%% lstlistings envvironment options start %%%%%%

\lstdefinestyle{customc}{ % C/C++ code snippet style
	belowcaptionskip = 1\baselineskip,
	breaklines       = true,
	frame            = L, % Double line on the left
	language         = C++,
	showstringspaces = false,
	basicstyle       = \ttfamily,
	keywordstyle     = \bfseries\color{green!40!black},
	stringstyle      = \color{orange},
	emphstyle        = \color{blue}, % Defined below
	tabsize          = 4,
	columns          = fullflexible,
	%NOTE The environment "lstlsiting" obviously can't know all types, especially custom defined ones. Because of this, non-primitive types names have to be added below for proper syntax highlighting.
	emph = {std, A, B, BDIT, Base, BaseOne, BaseTwo, C, CharT, Circle, Compr, ConstIterator, Cont, D, Deltoid, Derived, Derived2, DerivedLast, DerivedOne, DerivedTwo, Fact, Fiu, Foo, Greater, Hallgato, IT, InputIt, Iterator, Kor, Lany, Less, LessByX, LinAlgVector, List, Matrix, Negyszog, Negyzet, RAIT, Rombusz, S, Sielo, Sikidom, StringLength, T, T1, T2, Templ, U, Val, X, Y, Z, bad_cast, bidirectional_iterator_tag, const_iterator, deque, forward_iterator_tag, input_iterator_tag, iostream, istream, iterator, iterator_category, line_editor, list, map, multimap, multiset, ostream, pair, random_access_iterator_tag, set, string, strlen, strlenWrong, value_type, vector}
}

\lstdefinestyle{customasm}{ % Assembly code snippet style
	belowcaptionskip = 1\baselineskip,
	frame            = L, 
	language         = [x86masm]Assembler,
	basicstyle       = \ttfamily,
	commentstyle     = \itshape\color{yellow!40!black},
}

% A code style for C++ codes, which will be excluded by clang-format.
% Usefull eg. showing preprocessor directives, and multiple inclusion of the same file (those would be removed by clang-format)
\lstdefinestyle{customcpreproc}{%
  style = customc,
}

\lstset{
	escapechar = @,
	style      = customc, % Default code snippet style
	%NOTE In order to use special characters in code snippets, one has to manually define them.
	literate   = {á}{{\'a}}1 {é}{{\'e}}1 {í}{{\'i}}1 {ó}{{\'o}}1 {ú}{{\'u}}1	{Á}{{\'A}}1 {É}{{\'E}}1 {Í}{{\'I}}1 {Ó}{{\'O}}1 {Ú}{{\'U}}1	{ö}{{\"o}}1 {ü}{{\"u}}1 {Ö}{{\"O}}1 {Ü}{{\"U}}1
	{ű}{{\H{u}}}1 {Ű}{{\H{U}}}1 {ő}{{\H{o}}}1 {Ő}{{\H{O}}}1
	{€}{{\euro}}1 {£}{{\pounds}}1 {~}{$\sim$}{1}	
}

%%%%%%% lstlistings envvironment options end %%%%%%%

%%%%%%%% Compilation error fix begin %%%%%%%%
%NOTE https://tex.stackexchange.com/questions/376763/incomplete-iffalse-all-text-was-ignored-after-line-x-with-lstlisting
\makeatletter
\expandafter\let\csname active@char\string?\endcsname\relax
\expandafter\let\csname active@char\string!\endcsname\relax
\expandafter\let\csname active@char\string:\endcsname\relax

\initiate@active@char{?}
\initiate@active@char{!}
\initiate@active@char{:}
\makeatletter
%%%%%%%%% Compilation error fix end %%%%%%%%%

%%%%%%%% Custom command and environment definitions start %%%%%%%%
% Better looking left shift operator
\newcommand{\<}{<\hspace{0mm}<} 

% "note" environment definition
\theoremstyle{definition}
\newtheorem{note}{Megjegyzés}[subsection]
\newtheorem{exercise}{Feladat}[subsection]

% Commands for subfiles
\newcommand{\onlyinsubfile}[1]{#1} % Its argument appears only when the file is compiled on its own
\newcommand{\notinsubfile}[1]{} % Its argument appears only when the main file (or root file) is compiled
%%%%%%%%% Custom command and environment definitions end %%%%%%%%%